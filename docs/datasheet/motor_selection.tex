\chapter{Motor selection considerations}
When selecting a motor, it is important to consider its theoretical maximum rotational speed
(RPM\textsubscript{max}). This is derived from the motor's \emph{motor velocity constant}
(K\textsubscript{v}), its \emph{supply voltage} (V) and the controller's \emph{voltage utilization
factor} (F\textsubscript{util}). The RPM\textsubscript{max} is calculated as follows:

\[RPM\textsubscript{max} = K\textsubscript{v} \times V\textsubscript{supply} \times F\textsubscript{util}\]

K\textsubscript{v} is the ratio of the unloaded motor's rotational speed with the motor's peak
voltage (as measured across the coil wires). Its unit is revolutions per minute per volt (RPM/V)
or radians per volt second [rad/(V$\cdot$s)].

F\textsubscript{util} is a constant factor with a value between zero and one depending on the
PWM type. BLDC motor controllers with a conventional trapezoidal or six-step commutation always
have a factor of one. FOC-enabled motor controllers with SVPWM like Komar have a factor as follows:

\[F\textsubscript{util} = \frac{0.91}{\sqrt{3}}\]

For example, a motor with a \emph{motor velocity constant} (K\textsubscript{v}) of 320 RPM/V being
controlled by a Komar running on a fully charged 10S $\text{LiCoO}_\text{2}$ battery (i.e. 10 cells
of 4.2\,V) will have a theoretical maximum RPM of:

\[RPM\textsubscript{max} = 320 \times 10 \times 4.2 \times \frac{0.91}{\sqrt{3}} = 7061\]

In short, when selecting a motor for FOC ESCs like Komar, the motor's K\textsubscript{v} factor
must be no less than:

\[K\textsubscript{v} = \frac{RPM\textsubscript{max}}{V\textsubscript{DCmin}} \times \frac{0.91}{\sqrt{3}}\]

where:

\begin{enumerate}[label=(\roman*),leftmargin=\leftskip+1em,labelindent=!]
  \item RPM\textsubscript{max} is the desired maximum speed of the unload motor; and
  \item V\textsubscript{DCmin} is the minimum DC power supply voltage or the voltage of the
        fully discharged battery.
\end{enumerate}
